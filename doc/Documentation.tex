%Préambule du document :
\documentclass[11pt]{article}
\usepackage[utf8]{inputenc}
\usepackage[francais]{babel}

\usepackage{geometry}
\geometry{scale=0.75}

\usepackage{hyperref}
 
\title{Documentation}
\author{H4313}
\date{\today}

%Corps du document :
\begin{document}
\maketitle
\tableofcontents
\newpage

\section{Introduction}
Le nombre de véhicules a beaucoup augmenté ces dernières années. Les infrastructures routières se retrouvent alors de plus en plus encombrées. De ce fait, il est nécessaire de réfléchir à une gestion du trafic. De nos jours, les municipalités tentent d'améliorer les transports en commun et les modes de déplacements verts en incitant les automobilistes à ne plus prendre leurs véhicules pour leurs déplacements quotidiens. Cependant, certaines personnes n'ont pas le choix et doivent continuer à prendre leurs véhicules. Nous pensons par exemple aux personnes âgées, aux personnes à mobilité réduite ou bien encore aux livreurs. De plus, le nombre de bus pour les transports en commun participe à l'augmentation du trafic. La nécessité d'une optimisation de la gestion des feux de signalisation nous semblait évidente. C'est pourquoi nous avons décidé de nous pencher sur ce problème très complexe. En effet, plusieurs thèses ont été publiées sur ce sujet \footnote{Régulation du trafic urbain \url{http://www.biblio.univ-evry.fr/theses/2009/2009EVRY0035.pdf}} et l'optimum du problème n'a toujours pas été trouvé. Nous tenterons, dans le faible temps imparti, de pointer du doigt les points difficiles à l'optimisation des feux de signalisation et de montrer que les méthodes d'apprentissage peuvent être une bonne solution à ce problème.

\section{Simulateur}
Le Grand-Lyon ne possède pas dans son catalogue open-date de données sur les feux de signalisation. En effet, ceux-ci sont gérés par une société privée et la mise à disposition des données de ceux-ci n'est pas une option. Nous avons donc été contraint de développé un simulateur qui reflète du mieux que possible le comportement réel des usagers de la route.

\subsection{Modélisation des tronçons}
Afin de simplifier le problème, nous avons restreins la géométrie des carrefours à des carrefours où les tronçons sont à angles droits et ou chaque tronçon désert au maximum 3 directions (Gauche, Tout Droit, Droite). Chaque voie possède une ou plusieurs directions (Gauche, Tout Droit, Droite) ainsi qu'une vitesse maximale. 
Afin de rendre réaliste la trajectoire dans les intersections, nous avons choisi d'interpoler la trajectoire en virage par une courbe d'ordre 3. 
\subsection{Modèle de gestion du trafic}
Pour modéliser le comportement des véhicules, deux approches sont possibles. L’approche macroscopique, qui considère un flux de véhicule et effectue des calculs à l'instar de la mécanique des fluides. La deuxième approche, l'approche microscopique, considère les véhicules individuellement et tente de modéliser son comportement, éventuellement en fonction de celui des autres. 
Nous avons choisi d'utiliser une approche microscopique et de retenir le modèle du conducteur intelligent (Intelligent Driver Model en anglais \footnote{Page Wikipedia sur IDM \url{https://en.wikipedia.org/wiki/Intelligent_driver_model}}). Ce modèle est un des plus récents, un des plus efficaces, et un modèle dans lequel les équations restent compréhensibles et ont un sens physique.
Ce modèle donne la vitesse et l'accélération d'un véhicule en fonction de la vitesse maximale de la voie, de la distance avec l'obstacle le précédant et de différents paramètres comme l'accélération maximale ou la décélération maximale (freinage).

\subsection{Modèle du changement de voie}


\section{Apprentissage}

\section{Bilan du projet}
\subsection{infructueuses}


%% EXAMPLE CECI EST UN EXEMPLE
\subsection{Tentative 1}
\subsubsection{Qu'est ce que c'est ?}
\subsubsection{Pourquoi c'etait un echec ou une reussite ?}
%% CECI EST LA FIN DE L'EXEMPLE



\end{document}